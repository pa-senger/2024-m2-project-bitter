\documentclass[12pt]{article}
\usepackage{amsmath}
\usepackage[unicode=true, colorlinks=true, linkcolor=blue, urlcolor=cyan]{hyperref}
\usepackage{graphicx}
\usepackage{float}
\usepackage{caption}
\usepackage{listings}
\usepackage{xcolor}
\usepackage{pgfplots}
\usepackage{tikz}
\usetikzlibrary{snakes}
\usepackage{rotating}
\usetikzlibrary{arrows.meta, shapes}
\usepackage{pdflscape}   % Landscape pages
\usepackage[backend=biber,style=alphabetic]{biblatex} % Bibliography management
\addbibresource{references.bib} % Bibliography file
\graphicspath{{doc/}}

% Define colors
\colorlet{punct}{red!60!black}
\definecolor{background}{RGB}{240, 248, 255} % Pale Blue
\definecolor{delim}{RGB}{20,105,176}
\colorlet{numb}{magenta!60!black}

% Define JSON language
\lstdefinelanguage{json}{
    basicstyle=\ttfamily\footnotesize\color{black},
    basicstyle=\ttfamily\footnotesize\color{black},
    numbers=left,
    numberstyle=\scriptsize,
    stepnumber=1,
    numbersep=8pt,
    showstringspaces=false,
    breaklines=true,
    frame=lines,
    backgroundcolor=\color{background},
    literate=
     *{0}{{{\color{numb}0}}}{1}
      {1}{{{\color{numb}1}}}{1}
      {2}{{{\color{numb}2}}}{1}
      {3}{{{\color{numb}3}}}{1}
      {4}{{{\color{numb}4}}}{1}
      {5}{{{\color{numb}5}}}{1}
      {6}{{{\color{numb}6}}}{1}
      {7}{{{\color{numb}7}}}{1}
      {8}{{{\color{numb}8}}}{1}
      {9}{{{\color{numb}9}}}{1}
      {:}{{{\color{punct}{:}}}}{1}
      {,}{{{\color{punct}{,}}}}{1}
      {\{}{{{\color{delim}{\{}}}}{1}
      {\}}{{{\color{delim}{\}}}}}{1}
      {[}{{{\color{delim}{[}}}}{1}
      {]}{{{\color{delim}{]}}}}{1},
}



\setlength{\parskip}{1em}

\lstset{frame=single, showstringspaces=false, columns=fixed, basicstyle={\ttfamily}, commentstyle={\it}, numbers=left, tabsize=4}

\definecolor{codebackground}{RGB}{240, 248, 255}
\definecolor{codecomment}{RGB}{106,153,85}
\definecolor{codekeyword}{RGB}{30,30,255}
\definecolor{codestring}{RGB}{163,21,21}
\definecolor{codenumber}{RGB}{100,100,100}

\lstdefinestyle{modernstyle}{
    backgroundcolor=\color{codebackground},
    commentstyle=\color{codecomment},
    keywordstyle=\color{codekeyword},
    numberstyle=\tiny\color{codenumber},
    stringstyle=\color{codestring},
    basicstyle=\ttfamily\footnotesize\color{black},
    breakatwhitespace=false,
    breaklines=true,
    captionpos=b,
    keepspaces=true,
    numbers=left,
    numbersep=5pt,
    showspaces=false,
    showstringspaces=false,
    showtabs=false,
    tabsize=4
}


\lstset{style=modernstyle}

\begin{document}

\begin{titlepage}
\centering
\includegraphics[width=0.5\textwidth]{images/logo-ufr.png}\par\vspace{1cm}
\vspace{1.5cm}
{\huge\bfseries M2-Project: \\
Pre and Post-Processing applied to a LNCMI Bitter Magnet\par}
\vspace{2cm}
{\Large Pierre-Antoine Senger and Antoine Regardin\par}
\vfill


\vfill

% Bottom of the page
{\large Date: \today\par}
\end{titlepage}

\tableofcontents

\newpage

\section{Introduction}

\subsection{Context}

\subsection{Main objectives}

\subsection{Software and libraries}
To make the model of the bitter and mesh it, we will use Gmsh \cite{gmsh}, a three-dimensional
 finite element mesh generator with a built-in CAD engine and post-processor. 
Gmsh is an open-source software and is widely used in the scientific
community.

As for our simulations, our primary tool will be Feel++\cite{feelpp}, a robust and 
efficient open-source C++ library designed for solving partial differential equations using 
the finite element method\cite{fem}.



\section{Methodology}
\subsection{Geometry and mesh generation}

\subsection{Mathematical model}

\subsubsection{Physical parameters}

% \begin{landscape}
\begin{table}[h!]
	\centering
	\resizebox{\textwidth}{!}{ % Scale the table to the text width
	\begin{tabular}{l l l l l}
	\hline
	\textbf{Parameter} & \textbf{Symbol} & \textbf{Units} & \textbf{Baseline Value} & \textbf{Range or Distribution} \\
	\hline
	
	\hline
	\end{tabular}
	}
	\caption{Summary of key physical parameters for modeling the distribution of 
	temperature and current in the magnet.}
\end{table}
% \end{landscape}


\subsection{Discretization}



\section{Implementation}

\section{Complexity Analysis}

\section{Results}

\section{Prospects}

\section{Conclusion}

\newpage

\section{References}
\printbibliography
\end{document}